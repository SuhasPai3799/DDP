\section{Rudimant-Kompiler und Laufzeitsystem}

Der Rudimant-Kompiler übersetzt Regeldateien mit Extension \texttt{.rudi} in
Java-Dateien. Dazu braucht er eine Ontologie, in der die RDF Klassen und
Prädikate, die im \texttt{.rudi}-Code verwandt werden, spezifiziert sind.

Im Fall des POC liegen die Quelldateien in \texttt{src/main/rudi} und die
dazugehörende Ontologie in \texttt{src/main/resources/ontology}. Damit HFC
die Ontologie benutzen kann, muss sie im ntriples-Format vorliegen. Die
derzeitige Ontologie wird mit Protégé erstellt und aus dem OWL-XML Format
mit Hilfe des \texttt{rapper}-Tools in eine \texttt{.nt} ntriples Datei
übersetzt. \texttt{rapper} ist Teil des Ubuntu-Package \texttt{raptor2-utils},
das Script \texttt{ntcreate} im \texttt{poc} Verzeichnis updated alle nicht
aktuellen \texttt{.nt} Files aus den \texttt{.owl} Versionen.

Weitere settings, die für die Kompilation wichtig sind, finden sich in der
Datei \texttt{herbea.yml}, die von \texttt{compile} Skript benutzt wird. Auch
hier sind alle relativen Pfade relativ zum Verzeichnis, in dem die
\texttt{.yml} Datei liegt.

Für standalone Mockup-Tests kann der POC auch isoliert mit dem \texttt{run.sh}
script gestartet werden, die ``Sensordaten'' werden dann nach und nach in
der main-Methode eingespielt.

Der folgende Text ist leider unvollständig und muss noch ergänzt werden. Für
die meisten Konstrukte gibt es einige Beispiele in den \texttt{rudi}
Quelldateien.

\subsection{Rudimant Regeln}

TODO:
\begin{itemize}
\item ``Globale'' Funktionen und Variablen
\item RDF Zugriff, funktionale vs. relationale Prädikate
  \texttt{+=}, \texttt{-=}
\item Regeln und Labels
%\item \texttt{propose}
\item Dialogakte und\ {\Large\verb|^|}
\item Typinferenz
\item Überladene Vergleichsoperatoren und Tests
\item Funktionale Konstrukte (lambda)
\begin{verbatim}
boolean contains(Collection coll, Predicate pred);
boolean all(Collection coll, Predicate pred);
List<Object> filter(Collection coll, Predicate pred);
List<Object> sort(Collection coll, Comparator c);
\end{verbatim}
\item \texttt{import}
\end{itemize}

\subsection{Struktur des POC Rudimant-Projekts}

TODO: Siehe Bild für Softprak, Beschreibung von HerbeaAgent.rudi
vs. HerbeaAgent.java und Rolle von HerbeaClient

Die Basisklassen von Herbea sind \texttt{HerbeaClient}, der die Kommunikation
mit der Außenwelt herstellt, und \texttt{HerbeaAgent}, der Java-Funktionalität
zur Verfügung stellt, die sich nicht ohne weiteres in \texttt{rudi} Dateien
implementieren lässt (komplexe Queries an die Datenbank, etc.).

Zu \texttt{HerbeaAgent.java} gehört noch eine Datei \texttt{HerbeaAgent.rudi},
die sozusagen das Interface beschreibt, auf das der \texttt{rudi} Quellcode
zugreifen kann. Hier können auch statt der generischen Klasse \texttt{Rdf} die
Klassen aus der Ontologie spezifiziert werden, wenn diese genauer angegeben
werden können. Das hilft dem Kompiler bei der Typinferenz und dem richtigen
Zugriff mit RDF-Prädikaten.

\subsection{Default-Funktionalität im Laufzeitsystem}
Alles was in \texttt{Agent} bereitgestellt wird. Die aktuelle Liste der
bereitgestellten Funktionen finden sich in \texttt{rudimant} unter
\texttt{src/main/resources/Agent.rudi}.

\begin{itemize}
\item timeouts
\begin{verbatim}
void newTimeout(String name, int millis);
boolean isTimedOut(String name);
void removeTimeout(String name);
boolean hasActiveTimeout(String name);
\end{verbatim}
\item Senden von Dialogakten an die Generierung
\begin{verbatim}
DialogueAct emitDA(int delay, DialogueAct da);
DialogueAct emitDA(DialogueAct da);
\end{verbatim}
\item Zugriff auf DialogAkte aus der Session
\begin{verbatim}
// my last outgoing resp. the last incoming dialogue act
DialogueAct myLastDA();
DialogueAct lastDA();

// did i say something like ta in this session (subsumption)? If so, how many
// utterances back was it? (otherwise, -1 is returned)
int saidInSession(DialogueAct da);
// like saidInSession, only for incoming dialogue acts
int receivedInSession(DialogueAct da);

boolean waitingForResponse();
void lastDAprocessed();
\end{verbatim}
\end{itemize}

%%% Local Variables:
%%% mode: latex
%%% TeX-master: "master"
%%% End:
