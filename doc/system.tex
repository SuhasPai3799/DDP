\newcommand{\caret}{{\large\textbf{\textasciicircum}}}

\section{The \vonda Compiler}

The compiler turns the \vonda source code into Java source code using the
information in the ontology. Every source file becomes a Java class. The
generated code will not serve as an example of good programming practice, but a
lot of care has been taken in making it still readable and debuggable. The
compile process is separated into three stages: parsing and abstract syntax
tree building, type checking and inference, and code generation.

The \vonda compiler's internal knowledge about the program structure and the
RDF hierarchy takes care of transforming the RDF field accesses to reads from
and writes to the database. Beyond that, the type system, resolving the exact
Java, RDF or RDF collection type of arbitrary long field accesses,
automatically performs the necessary casts for the ontology accesses.

\subsection{\vonda's Architecture}

Figure~\ref{fig:architecture} shows the architecture of a runnable \vonda project.

\begin{figure}[htbp]
  \centering
  \tiny%
\begin{tikzpicture}[font=\bf\sffamily,
  box/.style={minimum width=4.5cm, minimum height=.35cm, draw=gray, text=black},
  gbox/.style={box, fill=lightgray},
  arr/.style={thick, -{Stealth}},
  larr/.style={thick, {Stealth}-},
  oarr/.style={thick, -{Stealth}},
  bbox/.style={box, fill=meddarkblue, text=ivory, draw},
  rbox/.style={node distance=0.3cm, font=\ttfamily},
  fbox/.style={node distance=4.2cm, font=\ttfamily},
  rlabel/.style={right, xshift=1mm, font=\sl},
  llabel/.style={left, xshift=-1mm, font=\sl}
]
\node[gbox](rmn) at (0,0.2) { Rule Module Class N };
\node[rbox, right= of rmn](rmnj){RuleModuleN.java};
\draw[larr] (rmnj) --  ++(1.5,0);
\node[fbox, right= of rmn](rmnr){RuleModuleN.rudi};
\draw[oarr] (rmnr) --  ++(-1.5,0);

\node[minimum width=4.5cm](ddd) at (0,-.2) { \normalsize ... };
\node[rbox, right= of ddd]{\small ...};
\node[fbox, right= of ddd]{\small ...};

\path (rmn) ++(0,-.75) node[gbox](rm1) { Rule Module Class 1 };
\node[rbox, right= of rm1](rm1j){RuleModule1.java};
\draw[larr] (rm1j) --  ++(1.5,0);
\node[fbox, right= of rm1](rm1r){RuleModule1.rudi};
\draw[oarr] (rm1r) --  ++(-1.5,0);

\path (rm1) ++(0,-.75) node[gbox](tlrc) { Top Level Rule Class };
\node[rbox, right= of tlrc](maj){MyAgent.java};
\draw[larr] (maj.east) ++(.35,0) --  ++(.72,0);
\node[fbox, right= of tlrc](mar){MyAgent.rudi};
\draw[oarr] (mar.west) -- ++(-.7,0);
\draw[arr] (tlrc.north) ++ (1,0) coordinate (tn) -- node[rlabel]{calls} (rm1.south -| tn);
\draw[arr] (rm1.south) ++ (-1,0) coordinate (rs) -- node[llabel]{imports} (tlrc.north -| rs);

\path (tlrc) ++(0,-.75) node[box, fill=ivory](caai) {Concrete Agent API Implementation};
\node[rbox, right= of caai]{MyAgentBase.java};
\draw[arr] (caai) -- node[rlabel]{extends} (tlrc);

\path (caai) ++(0,-.75) node[box, fill=darkblue, text=ivory](caid) {Common Agent API Interface Description}
 ++ (0,-.35) node[bbox](caii) {Common Agent API Implementation}
% ++ (0,-.35) node[bbox] {VOnDA Runtime}
 ++ (0,-.45) node[bbox, minimum height=.6cm](cb) {
\begin{minipage}{2cm}\centering Main event\\processing loop\end{minipage}
\begin{minipage}{2cm}\centering RDF Object Access\end{minipage}
}
 ++ (0, -.4) node[bbox] {Dialogue Act Creaton / Comparison}
 ++ (0,-0.35) node[bbox](nlp){
\begin{minipage}{2cm}\centering NLG\end{minipage}
\begin{minipage}{2cm}\centering NLU\end{minipage}
};
\node[rotate=90,yshift=2mm] at (cb.west){\scriptsize VOnDA Runtime};

\node[rbox, right= of caii]{Agent.java};
\node[rbox, right= of caid]{Agent.rudi};
\draw[arr] (caid) -- node[rlabel]{extends} (caai);
\draw (cb) ++(0,.3) -- +(0,-.51);
\draw (nlp) ++(0,.18) -- +(0,-.36);

\path (nlp) ++(0,-.75)
+(-1.125,0) node[fill=lightgray, minimum width=2.25cm, minimum height=.35cm]{}
+(1.125,0) node[fill=darkblue, text=ivory, minimum width=2.25cm, minimum height=.35cm]{}
+(0,0) node[box](ci) {\hspace{2.7ex}Client\color{ivory}\ \ Interface };
\node[rbox, right= of ci]{StubClient.java / MyClient.java};

\draw[arr] (ci.north) ++ (1,0) coordinate (ci1) -- (nlp.south -| ci1);
\draw[arr] (nlp.south) ++ (-1,0) coordinate (nl1) -- (ci.north -| nl1);

\node[minimum width=2cm, minimum height=.7cm, %draw,
      fill=meddarkblue, text=ivory, rotate=90](vc) at (5.3,-.6) {\bf\sffamily VOnDA Compiler};

\node[minimum width=1.3cm, minimum height=.8cm, %draw,
      fill=meddarkblue, text=ivory](dbg) at (7.3,-4.9)
      {\begin{minipage}{1.3cm}\centering\bf\sffamily VOnDA\\Debugger\end{minipage}};

% SUPER TEMPLATE FÜR DIE "TONNE"
\path [fill=midgray, text=black, draw=black] (5.3, -4)
   node[minimum height=1.24cm,minimum width=1.5cm] (db) {\cmp{1.3cm}{RDF\\Database}}
   ++(-.75,0.5)
   -- ++(0,-.9)
   arc [start angle=180, delta angle=180, x radius=.75cm, y radius=1.2mm]
   -- ++(0,.9)
   arc[start angle=0, delta angle=360, x radius=.75cm, y radius=1.2mm];

\draw[arr] (vc) --
  node[above, rotate=90, xshift=1.1mm]{Class \& Predicate}
  node[below, rotate=90, xshift=1.1mm]{Definitions} (db);
\draw[arr, {Stealth}-{Stealth}] (db.west)
       -- node[above]{Belief State Data} (db.west -| cb.east);

\draw[arr](dbg) -- (nlp);
\draw[arr](dbg) -- (db);
\draw[arr](dbg) -- (vc.south west);
\end{tikzpicture}

%%% Local Variables:
%%% mode: latex
%%% TeX-master: "vonda"
%%% End:

  \caption{Schematic of a \vonda interaction manager implementation}
  \label{fig:architecture}
\end{figure}

A \vonda project consists of an ontology, a custom extension of the abstract
\texttt{Agent} class (the so-called \emph{wrapper class}), a client interface
to connect the communication channels of the application to the agent, and a
set of rule files that are arranged in a tree, using \texttt{import}
statements. The blue core in Figure~\ref{fig:architecture} is the run-time
system that is part of \vonda, while all light grey elements are the
application specific parts of the agent. A \texttt{Yaml} project file contains
all necessary information for compilation: the ontology, the wrapper class, the
top-level rule file and other parameters, like custom compile commands for
\vonda's debugger.

The \vonda compiler translates rule files with the extension \texttt{.rudi} to
Java files. During this process, the ontology storing the RDF classes and
properties is used to automatically infer types, resolve whether field accesses
are actually accesses to the database, etc (see section
\ref{sec:typeinference}).
Every rule file can define variables and functions in \vonda syntax which are
then available to all imported files.

The current structure assumes that most of the Java functionality that is used
inside the rule files will be provided by the \texttt{Agent} superclass. There
are, however, alternative ways to use other Java classes directly (see section
\ref{sec:javatypes} for further info).  The methods and fields from the custom
wrapper class can be made available to all rule files by declaring them in the
interface connecting the \texttt{.rudi} code to the Java framework. This
interface must have the same name as the wrapper class but end with
\texttt{.rudi} (in the example of figure ~\ref{fig:architecture}, this would be
\texttt{MyAgent.rudi}).

\subsection{The \vonda Rule Language}
\label{sec:language}

\vonda's rule language looks very similar to Java/C++. There are a number of
specific features which make it much more convenient for the specification of
dialogue strageties. One of the most important features is the way objects in
the RDF store can be used throughout the code: RDF objects and classes can be
treated similarly to those of object oriented programming languages, including
the type inference and inheritance that comes with type hierarchies.

\subsubsection{The Structure of a \vonda File}

\vonda does not require to group statements in some kind of high-level
structure like e.g. a class. It is, in fact, not possible to define classes in
\texttt{.rudi} files at all, rules and method declarations have to be put
directly into the rule file. The same holds for every kind of valid
(Java-) statement, like assignments, \texttt{for} loops etc. From this, the
compiler will create a Java class where the methods and rules that are
transformed are represented as methods of this specific (generated)
class. All other statements as well as auto-generated calls to the methods
representing the rules will be put into the \texttt{process()} method that
\vonda creates to build a rule evaluation cycle. In doing so, the execution
order of all statements, including the rules, is preserved.

This functionality offers possibilities to e.g. define and process high-level
variables that you might want to have access to in subsequent rules or to insert
termination conditions that prevent some rule executions.

\textbf{Warning:} It is important to know that variables declared globally
in a file will be transformed to fields of the Java class. We found that in
very rare occasions, this can lead to unexpected behaviour when using them in a
propose or timeout block as well as changing them in a global statement. As
proposes and timeouts will not immediately be executed, they need every
variable used inside them to be effectively final. \vonda leaves the evaluation
of validness of variables for such blocks to Java. We found that Java might
mistakenly accept variables that are not effectively final, what might lead to
completely unexpected behaviour when proposes and timeouts with changed
variable values are executed.

The only important exeption, where globally defined variables persist throughout
the whole runtime of the system are variables defined in the top-level rule
file. This is on purpose, and can be used to define persistent
variables also usable in lower-level rule files.

\subsubsection{RDF accesses and functional vs. relational properties}
\label{sec:rdfaccesses}

\begin{figure}[htb]
\rule{7mm}{0pt}\begin{minipage}{0.45\columnwidth}
\small%
\begin{lstlisting}[numbers=left,numberstyle=\scriptsize]
user = new Animate;
user.name = "Joe";
set_age:
if (user.age <= 0) {
  user.age = 15;
}
\end{lstlisting}
\end{minipage}\vrule\hspace{1ex}
\begin{minipage}{0.44\columnwidth}
    \small\begin{tikzpicture}[
  blob/.style={circle, fill=yellow!50!white, minimum width=2mm},
  txt/.style={node distance=3mm}]
  \draw (0,0) node (agent) [blob]{};
  % BEWARE: RIGHT OF= IS DEPRECATED, DON'T USE IT
  \node (agtxt) [right= 0.1 of agent, txt] {Agent};
  \node (name) [below= 0.4 of agtxt.west, anchor=west]{\emph{name}: \texttt{xsd:string}};
  \node (animate) [blob, below=0.7 of agtxt.west]{};
  \node (antxt) [right= 0.1 of animate, txt] {Animate};
  \node (name) [below= 0.4 of antxt.west, anchor=west]{\emph{age}: \texttt{xsd:int}};
  \node (inanimate) [blob, below= 0.5 of animate, node distance=9mm]{};
  \node (intxt) [right= 0.1 of inanimate, txt] {Inanimate};
  \draw (agent) |- (animate);
  \draw (agent) |- (inanimate);
\end{tikzpicture}
\end{minipage}
  \caption{Ontology and \vonda code}
  \label{fig:rdfobjects}
\end{figure}

Figure \ref{fig:rdfobjects} shows an example of \vonda code, and how it relates
to RDF type and property specifications, schematically presented on the right.
The domain and range definitions of properties are picked up by the compiler
and then used in various places, e.g., to infer types, do automatic code or
data conversions, or create ``intelligent'' boolean tests, like in line 4,
which will expand into two tests, one testing for the existence of the property
for the object, and in case that succeeeds, a test if the value is smaller or
equal than zero. If there is a chain of more than one field/property access,
every part is tested for existence in the target code, keeping the source code
as concise as possible (see also table~\ref{table:multi-predaccess}). Also for
reasons of brevity, the type of a new variable needs not be given if it can be
inferred from the value assigned to it.

New RDF objects can be created with \texttt{new}, similar to Java objects; they
are immediately reflected in the database, as are all changes to already
existing objects.

\begin{table}[htbp]
\small\begin{minipage}{0.5\textwidth}
\begin{lstlisting}
c = new Child;
nm = c.name;
c.name = "new name";
Set middle = c.middleNames;
c.middleNames += "John";
c.middleNames -= "James";
c.middleNames = null;
\end{lstlisting}
\end{minipage}{\LARGE$\Rightarrow$}\rule{6.4cm}{0pt}\\
\begin{minipage}{0.99\textwidth}
\begin{lstlisting}
c = palAgent._proxy.getClass("<dom:Child>").getNewInstance(palAgent.DEFNS);
nm = c.getString("<upper:name>");
c.setValue("<upper:name>", "new name");
middle = c.getValue("<upper:middleNames");
c.getValue("<upper:middleNames>").add("John");
c.getValue("<upper:middleNames>").remove("James");
c.clearValue("<upper:middleNames>");
\end{lstlisting}
\end{minipage}
  \caption{Examples for an RDF property access}
  \label{tab:property-access}
\end{table}

To be able to \emph{parametrize} the field access to RDF objects, \vonda has a
special mechanism which can lead to subtle and hard-to-find programming errors,
if used wrongly. If, for example, a string variable with the name
\texttt{middleNames} would be available in the scope of the above \vonda source
code, the resulting code would have looked like in table~\ref{tbl:varpropacc}
instead, using the \emph{value} of the variable to determine the field
name. For the given code, this does not make sense, but it can be useful in
situations where, e.g., the property to use is determined by the input
utterance.

\begin{table}[htbp]
\small\begin{minipage}{0.99\textwidth}
\begin{lstlisting}
// somewhere before in the scope ...
String middleNames = "<upper:middleNames>";
...
middle = c.getValue(middleNames);
c.getValue(middleNames).add("John");
c.getValue(middleNames).remove("James");
c.clearValue(middleNames);
\end{lstlisting}
\end{minipage}
\caption{RDF property access using the value of a variable as property name}
  \label{tab:varpropacc}
\end{table}

The connection of \vonda to the ontology loaded into HFC during compile time
enables the compiler to recognize the correct RDF class to create a new
instance when encountering \texttt{new}, and to resolve field/property
accessses to all RDF instances. Field accesses as shown in line 2 and 3 of
table \ref{tab:property-access} will be analyzed and transformed into database
accessses. \vonda will also draw type information from the database. If the
name property of the RDF class \texttt{Child} is of type \texttt{String},
exchanging line 2 by the line \texttt{int name = c.name} will result in a
warning of the compiler. During this process, the compiler will automatically
also use the correspondence of XSD and Java types shown in table
\ref{fig:RdfToJava}.

\begin{table}[htb]
{\small\ttfamily\begin{center}
\begin{tabular}{ll@{\hspace{8em}}ll}
<xsd:int> & Integer & <xsd:integer> & Long\\
<xsd:string> & String & <xsd:byte> & Byte\\
<xsd:boolean> & Boolean & <xsd:short> & Short\\
<xsd:double> & Double & <xsd:dateTime> & Date\\
<xsd:float> & Float & <xsd:date> & XsdDate\\
<xsd:long> & Long & <xsd:dateTimeStamp> & Long
\end{tabular}\end{center}}\vspace{-2ex}
\caption{\label{fig:RdfToJava}Standard RDF types and the Java types as which they will be recognized}
\end{table}

Moreover, \vonda determines whether an access is made using functional or
relational predicates and will handle it accordingly, assuming a collection
type if necessary. In the rule language, the operators \texttt{+=} and
\texttt{-=} are overloaded. They can be used with sets and lists as shortcuts
for adding and deleting objects. \texttt{a += b} will be compiled to
\texttt{a.add(b)} and \texttt{a -= b} results in \texttt{a.remove(b)}, as shown
in table~\ref{tab:property-access}.

\subsubsection{Rules and rule labels}

The core of \vonda dialogue management are the dialogue rules, which will be
evaluated at run-time system on every trigger generated from the environment or
the internal processing.
A rule (optionally) starts with a name that is given as a Java-like label: an
identifier followed by a colon. Following this label is an
\texttt{if}-statement, with optional \texttt{else} case. The clause of the
\texttt{if}-statement expresses the condition under which the rule, or rather
the \texttt{if} block, is to be executed; in the \texttt{else} block you can
define what should happen if the condition is \texttt{false}, like stopping the
evaluation of (a sub-tree of) the rules if necessary information is missing.

\begin{figure}[htb]
\begin{small}
\begin{lstlisting}
introduction:
  if (introduction) {
    if (user.unknown) {
      ask_for_name:
        if (talkative) askForName();
    } else
      greetUser();
  }
\end{lstlisting}
\end{small}\vspace{-2ex}
\caption{A simple rule}
\end{figure}

Rules can be nested to arbitrary depth, so \texttt{if}-statements inside a rule
body can also be labelled. The labels are a valuable tool for debugging the
system at run-time, as they can be logged live with the debugger GUI
(cf. chapter \ref{sec:debugger}). The debugger can show you which rules were
executed when and what the individual results of each base clause of the
conditions were.

\subsubsection{The \texttt{propose} and \texttt{timeout} constructs}

There are two statements with a special syntax and semantics: \texttt{propose}
and \texttt{timeout}. \texttt{propose} is \vonda's current way of implementing
probabilistic selection. All (unique) propose blocks that are in active rule
actions are collected, frozen in the execution state in which they were
encountered, like closures known from functional programming languages. When
all possible proposals have been selected, a statistical component decides
on the ``best'', whose closure is then executed.

\begin{figure}[h]
  \centering\small%
\begin{lstlisting}
if (!saidInSession(#Salutation(Meeting)) {
  // Wait 7 secs before taking initiative
  timeout("wait_for_greeting", 7000) {
    if (! receivedInSession(#Greeting(Meeting))
      propose("greet") {
        da = #InitialGreeting(Meeting);
        if (user.name) da.name = user.name;
        emitDA(da);
      }
  }

  if (receivedInSession(#Salutation(Meeting))
    propose("greet_back") { // We assume we know the name by now
      emitDA(#ReturnGreeting(Meeting, name={user.name});
    }
  }
}
\end{lstlisting}\vspace*{-3ex}
  \caption{\texttt{propose} and \texttt{timeout} code example}
  \label{fig:propose}
\end{figure}

\texttt{timeout}s generate the same kind of closures, but with a different
purpose. They can for example be used to trigger proactive behaviour, or to
check the state of the system after some amount of time, or in regular
intervals. A timeout will only be created if there is no active timeout with
the same name, otherwise, if the time delay is different than that of the last
\texttt{timeout} call, the delay will be set to the new value. For special
needs, the functions in table~\ref{tbl:timeoutfns} are useful to achieve
specific behaviours based on \texttt{timeout}s.

\begin{table}[htb]
\begin{tabular}{lp{.65\textwidth}}
\texttt{isTimedOut(\emph{label})}& returns \texttt{true} if a timeout with that
  label fired. This can be reset only by calling
  \texttt{removeTimeout(\emph{label})}, and is especially convenient to
  implement timeouts that should only be triggered once in a session. \\
\texttt{removeTimeout(\emph{label})}& see \texttt{isTimedOut(\emph{label})}\\
\texttt{cancelTimeout(\emph{label})}& cancels an \emph{active} timeout if there
                                      is one, has no effect otherwise \\
\texttt{hasActiveTimeout(\emph{label})}& returns true if there is an
  active timeout with that label \\
\end{tabular}
\caption{\label{tbl:timeoutfns}Functions for fine tuning \texttt{timeout} behaviour}
\end{table}

There are two variants of \texttt{timeout}: \emph{labeled timeouts}, like the
one in the previous example which run out after the specified time (unless they
are cancelled before running out) and then execute their body, and
\emph{behaviour timeouts}, where the first argument is a dialogue act (see next
section) instead of a label. These are executed either when the specified time
is up or the behaviour that was triggered by the dialogue act is finished
(e.g. the audio generated by a text-to-speech engine ended, or a specified
motion came to an end), whatever comes first.

The following code patterns may help to use the different possibilities that
timeouts offer:

{\small%
\begin{lstlisting}
// timeout triggered exactly once per session
if (! hasActiveTimeout("robot_starts") && ! isTimedOut("robot_starts"))
  timeout("robot_starts", 4000) { ... }

// timeout reoccurring every 1000 milliseconds
if (! hasActiveTimeout("reptimeout"))
  timeout("reptimeout", 1000) { ... }

// ensure that something happens even if the expected condition does not
// become true after 10 seconds
if (! condition && ! hasActiveTimeout("ensure_cond")) {
  timeout("ensure_cond", 10000) {
    if (! condition) {
      // clean up
    }
  }
}
\end{lstlisting}}

\subsubsection{Interrupting the rule evaluation cycle}

There are multiple ways to stop rule evaluation locally (i.e. skipping the
evaluation of the current subtree) or globally (i.e. stopping the whole
evaluation cycle).
You can skip the evaluation of a specific rule you are currently in with the
statement \texttt{break label\_name;}. This will only stop the rule with the
respective label (no matter how deep the break statement is nested in it), such
that the next following rule is evaluated next.

If the evaluation is cancelled with the keyword \texttt{cancel}, all of the
following rules in the current file will be skipped (including any imported
rules). If the keyword \texttt{cancel\_all} is used, none of the following
rules, neither local nor higher in the rule tree, will be evaluated. This is
the \vonda way of deciding to not further evaluate whatever triggered the
current evaluation cycle and will mostly be used as an 'emergency exit', as the
dialogue rules should be rejecting any non-matching trigger by themselves.

To leave \texttt{propose} and \texttt{timeout} blocks, you need to use an empty
\texttt{return}, as they are only reduced representations of normal function
bodies.

\todo[inline]{There should be a complete description of the rule evaluation
  cycle, with details such as the initialization of rule class objects, etc.}

\subsubsection{Dialogue Acts}
\label{sec:caret}

A central functionality of a dialogue system is receiving and emitting dialogue
acts that result from a user utterance resp. can be transformed to natural
language by a generation component to communicate with the user. In \vonda,
the function for sending dialogue acts is called \texttt{emitDA}.

The dialogue act representation is an internal feature of \vonda. We are
currently using the DIT++ dialogue act hierarchy \citep{bunt2012iso} and
shallow frame semantics along the lines of FrameNet
\citep{ruppenhofer2016framenet} to represent dialogue acts. The natural
language understanding and generation units connected to \vonda should
therefore be able to generate or, respectively, process this representation.

\begin{figure}[htb]
  \centering\small\texttt{emitDA(\#Inform(Answer, what=\{solution\}));}
  \vspace*{-1ex}\caption{\label{fig:DA}Dialogue Act Example}
\end{figure}

Figure \ref{fig:DA} shows the dialogue act representation in \vonda, as passed
to, e.g., the \texttt{emitDA} function. \texttt{Inform}\verb|(...)| will be
recognized by \vonda as dialogue act because it has been marked with
\verb|#|. It will then create a new instance of the class DialogueAct that
contains the respective modifications. As a default, arguments of a DialogueAct
creation (i.e., character strings on the left and right of the equal sign) are
seen as and transformed to constant (string) literals, because most of the time
that is what is needed.  Surrounding a character sequence with curly brackets
(\texttt{\{\}}) marks it as an expression that should be evaluated. In fact,
arbitrary expressions are allowed inside the curly brackets, and converted
automatically to a string, if necessary and possible.

While this kind of shallow semantics is enough for many applications, we
already experience its shortcomings when trying to handle, for example, social
talk. One of the next improvements will be the extension of Dialogue Acts to
allow for embedded structures.


\subsubsection{Type inference and overloaded operators}
\label{sec:typeinference}

\vonda allows static type assignments and casting, but in many cases these can
be avoided. If, for example, the type of the expression on the right-hand side
of a declaration assignment is known or inferrable, it is not necessary to
explicitely state it.

You can also declare variables final.

\begin{table}[htbp]
  \begin{small}
\begin{minipage}{.55\textwidth}
\begin{lstlisting}
if (! c.user.personality.nonchalance){ ... }
\end{lstlisting}
\end{minipage}\rule{2cm}{0pt}{\LARGE$\Rightarrow$}\hfill\\
\begin{lstlisting}
if (!((((c != null) && (c.user != null)) && (c.user.personality != null))
      && (c.user.personality.nonchalance != null))) {
  ...
}
\end{lstlisting}\end{small}\vspace*{-2ex}
\caption{Transformation of complex boolean expressions}
\label{tab:multi-predaccess}
\end{table}
\vspace*{10pt}

A time-saving feature of \vonda which also improves readability is the
automatic completion of boolean expressions in the clauses of \texttt{if},
\texttt{while} and \texttt{for} statements. As it is obvious in these cases
that the result of the expression must be of type boolean. \vonda automatically
fills in a test for existence if it is not. When encountering field accesses,
it makes sure that every partial access is tested for existence (i.e., not
\texttt{null}) to avoid a \texttt{NullPointerException} in the runtime
execution of the generated code.

Be aware that the expansion in the figure only occurs if the multiple field
access is used as boolean test. In the following example, the first clause in
the boolean expression should not be omitted, since a
\texttt{NullPointerException} could still occur because the second clause does
not trigger an automatic test for existence of the \texttt{status} of
\texttt{activity}:

\begin{lstlisting}
if (activity.status && activity.status == "foo"){ ... }
\end{lstlisting}

Many operators are overloaded, especially boolean operators such as
\textbf{\texttt{<=}}, which compares numeric values, but can also be used to
test if an object is of a specific class, for subclass tests between two
classes, and for subsumption of dialogue acts.

\begin{table}[htbp]
\centering
{\footnotesize%
\begin{minipage}{0.28\textwidth}
\begin{lstlisting}
if (sa <= #Question){
  ...
}
\end{lstlisting}
\end{minipage}\vline\hspace{1em}
\begin{minipage}{0.6\textwidth}
\begin{lstlisting}
if (sa.isSubsumedBy(new DialogueAct("Question")) {
  ...
}
\end{lstlisting}
\end{minipage}}\vspace*{-2ex}
\caption{\label{tab:overloaded-comparison}Overloaded comparison operators}
\end{table}

\subsubsection{External methods and fields}
\label{sec:javatypes}

As mentioned before, you can use every method or field you declare in your
custom Agent implementation in your \vonda code. Their declaration in the
Java-rudi interface should look like a normal Java field or method definition
(cfg. table~\ref{tab:javadef}). It is possible to use generics in these
definitions, although they are, for complexity reasons, restricted to be one
single uppercase letter.

\begin{table}[htbp]
\small
\begin{lstlisting}
MyType someVariable;          // field of class MyType
MyType someFun(ClA a, ClB b); // method someFun with signature
                              // (ClA, ClB) --> MyType
                              // return type can be void for a procedural method
\end{lstlisting}\vspace*{-2ex}
\caption{\label{tab:javadef}Defintions of existing Java fields and methods for
  \vonda}
\end{table}

There is a variety of standard Java methods called on Java classes that \vonda
automatically recognizes, like e.g. the \texttt{substring} method for
Strings. If you find that you need \vonda to know the signature of a new method
or a field that is defined in a specific class, you can provide \vonda with
knowledge about them by adding their definition to the interface as follows:

\begin{table}[htbp]
\centering
\small
\begin{lstlisting}
[SomeClass]. myType Function(typeA a); // declaration of SomeClass method
[SomeClass]. myType someVar;           // declaration of SomeClass field
[List<T>]. T get(int a);               // use of Generics is possible and
                                       // used in type inference
\end{lstlisting}\vspace*{-2ex}
\caption{\label{tab:methoddef}Definition of a non-static method of Java objects}
\end{table}

It is important to realize that all declarations in the interface are only
compile time information for \vonda and will not be transferred to the compiled
code, whereas declarations in the rule code itself will also appear in the
compiled code.

\subsubsection{Functional constructs}

\vonda allows to specify \texttt{Function} arguments, where lambda
constructions can then be used in the code. Currently, the functions listed in
table \ref{tab:lambda-functions} are pre-defined in the \texttt{Agent} class.
If you for example want to filter a set of RDF objects by a subtype relation,
you can write:
{\small\begin{lstlisting}
des = filter(agent.desires, (d) -> (d <= UrgentDesire));
\end{lstlisting}}

\begin{table}[htbp]
\small%
\begin{lstlisting}
boolean some(Collection<T> coll, Function<Boolean, T> pred);
boolean all(Collection<T> coll, Function<Boolean, T> pred);
List<T> filter(Collection<T> coll, Function<Boolean, T> pred);
List<T> sort(Collection<T> coll, Function<Integer, T, T> comp);
Collection<T> map(Collection<S> coll, Function<T, S> f);
int count(Collection<T> coll, Function<Boolean, T> pred);
T first(Collection<T> coll, Function<Boolean, T> pred);
\end{lstlisting}\vspace*{-2ex}
\caption{\label{tab:lambda-functions}Functions that take lambda expressions as an argument}
\end{table}


\subsubsection{\texttt{import}}

With the \texttt{import} statement, e.g. \texttt{import File;}, which needs to
appear at the root level of rule files, the rules and definitions in
\texttt{File.rudi} and its imported files are included at the position of the
\texttt{import}.

This inclusion has two important effects. On the one hand, it triggers the
compilation of the included file at exactly this point, such that any fields
and methods known at this time will be available in the imported file. On the
other hand, all the rules contained in the imported file will be inserted in
the rule cycle at the specific position of the \texttt{import}, that is, in the
resulting code the \texttt{process()} method of the imported file will be
executed.

\texttt{import} makes it possible to organize the rules and local declarations
of a project into meaningful sub-units. This supports modularity, as different
subtrees of the \texttt{import} hierarchy can easily be added, moved, taken
away or re-used in different projects.

\subsubsection{Java-Code verbatim in rule files} \label{sec:rudi-verbatim}

To maintain simplicity, \vonda intentionally only provides limited Java
functionalities. Whatever is not feasible in \vonda source code should be done
in methods in the wrapper class or other helper classes.

In cases where this is not possible and you urgently need a functionality of
Java that \vonda cannot parse or represent correctly, you can use the verbatim
inclusion feature. Everything between \verb|/*@| and \verb|@*/| will be treated
like a multi-line Java comment, meaning the content is not parsed or evaluated
further. It will be transferred to the compiled code as is into the intended
position.

In particular, this functionality can be used to import Java classes (i.e.,
with Java import statements) at the beginning of a rule file. You should
however be aware that \vonda will not know that these classes have been
imported, nor their methods and fields. It will however accept creations of
instances of unknown classes, as well as your casting of results of unknown
methods. If you want \vonda to have type information about methods called on
instances on one of these classes, you can put this information into the type
interface of the wrapper class (cfg. chapter \ref{sec:javatypes}).

\section{The Run-Time System}
%\subsection{Run-Time Library}
\todo{AW: This is just a different language than the rest of the documentation... reformulate?}

The run-time library contains the basic functionality for handling the rule
processing, including the proposals and timeouts, and for the on-line
inspection of the rule evaluation. There is, however, no blueprint for the main
event loop, since that depends heavily on the host application. The run-time library also
contains methods for the creation and modification of shallow semantic
structures, and especially for searching the interaction history for specific
utterances. Most of this functionality is available through the abstract
\texttt{Agent} class, which has to be extended to a concrete class for each
application.

There is functionality to talk directly to the HFC database using queries (compare section \ref{sec:hfc_usage}), in
case the object view is not sufficient or to awkward.
% The natural language
%understanding and generation components can be exchanged by implementing existing
%interfaces, and the statistical component is connected by a message exchange
%protocol. A simple generation engine based on a graph rewriting module is
%already integrated, and is used in our current system as a template based
%generator. The example application also contains a VoiceXML based
%interpretation module.

The set of your reactive \vonda rules is executed whenever there is a change in the
information state (IS). These changes are caused by incoming sensor or
application data, intents from the speech recognition, or expired timers.
%Rules are labeled if-then-else statements, with complex conditions and shortcut
%logic, as in Java or C. The compiler analyses the base terms and stores their
%values during processing for dynamic logging.
A rule can have direct effects,
like changes in the IS, or system calls. Furthermore, it can generate so-called
\emph{proposals}, which are (labeled) blocks of code in a frozen state that
will not be immediately executed, similar to closures.

All rules are repeatedly applied until a fix point is reached: No new proposals
are generated and there is no IS change in the last iteration. Then, the set of
proposals is evaluated by a statistical component, which will select the best
alternative. This component can be exchanged to make it as simple or elaborate
as necessary, taking into account arbitrary features from the data storage.

\subsection{Functionalities (methods) Provided by the Run-Time System}
%Alles was in \texttt{Agent} bereitgestellt wird. Die aktuelle Liste der
%bereitgestellten Funktionen finden sich in \texttt{rudimant} unter
%\texttt{src/main/resources/Agent.rudi}.
The following methods are declared in \texttt{src/main/resources/Agent.rudi}; their implementation is provided by Java itself or the \vonda framework.

\vspace*{2ex}

\newcommand{\pgr}[1]{\newline\noindent\textbf{#1}}

\pgr{Pre-added Java methods}
\begin{small}
\begin{lstlisting}
[Object]. boolean equals(Object e);
[String]. boolean startsWith(String s);
[String]. boolean endsWith(String s);
[String]. String substring(int i);
[String]. String substring(int begin, int end);
[String]. boolean isEmpty();
[String]. int length();

[List<T>]. T get(int a);
[Collection<T>]. void add(Object a);
[Collection<T>]. boolean contains(Object a);
[Collection<T>]. int size();
[Collection<T>]. boolean isEmpty();
[Map<S, T>]. boolean containsKey(S a);
[Map<S, T>]. T get(S a);
[Array<T>]. int length;
\end{lstlisting}
\end{small}

\pgr{Short-hand conversion methods from Agent}
\begin{small}
\begin{lstlisting}
int toInt(String s);
float toFloat(String s);
double toDouble(String s);
boolean toBool(String s);
String toStr(T i);  // T in (int, short, byte, float, double, boolean)
\end{lstlisting}
\end{small}

\pgr{Other Agent methods}
\begin{small}
\begin{lstlisting}
// Telling the Agent that something changed
void newData();

String getLanguage();

// Random methods
int random(int limit); // return and int between zero and limit (excluded)
float random();        // return a random float between zero and one (excluded)
T random(Collection<T> coll); // select a random element from the collection

long now();     // return the current time since the epoch in milliseconds

Logger logger;  // Global logger instance

// discarding actions and shutdown
void clearBehavioursAndProposals();
void shutdown();
\end{lstlisting}
\end{small}

\pgr{Timeouts}
\begin{small}
\begin{lstlisting}
void newTimeout(String name, int millis);
boolean isTimedOut(String name);
void removeTimeout(String name);
boolean hasActiveTimeout(String name);
// cancel and remove an active timeout, will not be executed
void cancelTimeout(String name);
\end{lstlisting}
\end{small}

\pgr{Methods dealing with dialogue acts}
\begin{small}
\begin{lstlisting}
// sending of dialogue acts
DialogueAct createEmitDA(DialogueAct da);
DialogueAct emitDA(int delay, DialogueAct da);
DialogueAct emitDA(DialogueAct da);
[DialogueAct]. String getDialogueActType();
[DialogueAct]. void setDialogueActType(String dat);
[DialogueAct]. String getProposition();
[DialogueAct]. void setProposition(String prop);
[DialogueAct]. boolean hasSlot(String key);
[DialogueAct]. String getValue(String key);
[DialogueAct]. void setValue(String key, String val);
[DialogueAct]. long getTimeStamp();

// Access to dialogue acts of the current session
// my last outgoing resp. the last incoming dialogue act
DialogueAct myLastDA();
DialogueAct lastDA();

// Did I say something like ta in this session (subsumption)? If so, how many
// utterances back was it? (otherwise, -1 is returned)
int saidInSession(DialogueAct da);
// like saidInSession, only for incoming dialogue acts
int receivedInSession(DialogueAct da);

// Check if we asked a question that is still pending
boolean waitingForResponse();
// Mark last incoming DA as treated and not pending anymore (stop rules firing)
void lastDAprocessed();

DialogueAct addLastDA(DialogueAct newDA);
[DialogueAct]. void setProposition(String prop);
\end{lstlisting}
\end{small}

\pgr{Functions allowing lambda expressions (functional arguments)}
\begin{small}
\begin{lstlisting}
boolean some(Collection<T> coll, Function<Boolean, T> pred);
boolean all(Collection<T> coll, Function<Boolean, T> pred);
List<T> filter(Collection<T> coll, Function<Boolean, T> pred);
List<T> sort(Collection<T> coll, Function<Integer, T, T> c);
Collection<T> map(Collection<S> coll, Function<T, S> f);
int count(Collection<T> coll, Function<Boolean, T> pred);
T first(Collection<T> coll, Function<Boolean, T> pred);
\end{lstlisting}
\end{small}

\pgr{Methods on \texttt{Rdf} and \texttt{RdfClass} objects}
\begin{small}
\begin{lstlisting}
Rdf toRdf(String uri);
[Rdf]. String getURI();
[Rdf]. boolean has(String predicate);
[Rdf]. long getLastChange(boolean asSubject, boolean asObject);

RdfClass getRdfClass(String s);
boolean exists(Object o);

// return only the name part of an URI (no namespace or angle brackets)
String getUriName(String uri);
\end{lstlisting}
\end{small}


%%% Local Variables:
%%% mode: latex
%%% TeX-master: "userguide"
%%% End:
